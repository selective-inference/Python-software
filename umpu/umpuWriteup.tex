\documentclass{article}


\usepackage{graphicx,natbib}
\usepackage{../Stats/LaTeX/mycommands}
\usepackage{amsmath,amssymb,amsthm,bm,enumerate}
\usepackage{latexsym,color,verbatim,minipage-marginpar,caption,multirow}

\begin{document}


\newtheorem{theorem}{Theorem}
\newtheorem{corollary}[theorem]{Corollary}
\newtheorem{lemma}[theorem]{Lemma}
\newtheorem{observation}[theorem]{Observation}
\newtheorem{proposition}[theorem]{Proposition}
\newtheorem{definition}[theorem]{Definition}
\newtheorem{claim}[theorem]{Claim}
\newtheorem{fact}[theorem]{Fact}
\newtheorem{assumption}[theorem]{Assumption}

\renewcommand{\S}{\mathcal{S}}
\newcommand{\FDR}{\textnormal{FDR}}
\newcommand{\FCR}{\textnormal{FCR}}

\title{UMAU Interval Computation}
\maketitle

\newcommand{\crt}{\rho}

\begin{abstract}
  This document explains how to compute UMPU rejection regions for a
  truncated Gaussian distribution, and invert them to obtain UMAU
  confidence intervals for $\mu$.
\end{abstract}

Let $X\sim \L(N(\mu, \sigma^2) \gv X \in S)$, for $S = \bigcup_{k=1}^K
(a_k, b_k)$.  Without loss of generality, assume $\sigma^2 = 1$.

Write
\begin{equation}
  p_k(\mu) = \Phi(b_k-\mu) - \Phi(a_k-\mu), \;\;\; p(\mu) =
  \sum_{k=1}^K p_k(\mu)
\end{equation}

The density function of $X$ is
\begin{equation}
  f_\mu(x) = p(\mu)^{-1} \phi(x-\mu) \1\{x\in S\},
\end{equation}
which is a one-parameter exponential family with sufficient statistic
$x$ and natural parameter $\mu$.  Then, according to Lehmann \&
Romano, the UMPU acceptance region for $H_0:\,\mu_0$ is an interval $(c_1, c_2)$,
satisfying
\begin{align}
  \P_{\mu_0}(c_1 \leq X \leq c_2) &= 1-\alpha\\
  \E_{\mu_0}(X; c_1 \leq X \leq c_2) &= (1-\alpha) \E_{\mu_0} X
\end{align}
which can be written as
\begin{align}
  \int_{(c_1,c_2) \cap S} \phi(x-\mu)\,dx &= (1-\alpha) \int_S
  \phi(x-\mu)\,dx = (1-\alpha)p, \\
  \int_{(c_1,c_2) \cap S} x\phi(x-\mu)\,dx &= (1-\alpha) \int_S
  x\phi(x-\mu)\,dx = m(\mu)
\end{align}

The first thing we do is to write $c_2$ implicitly as a function of
$c_1$ (for $F_\mu(c_1) \leq \alpha$):
\begin{equation}
  c_2(c_1) = F_\mu^{-1}(F_\mu(c_1) + 1-\alpha)
\end{equation}
Note that $c_2'(c_1) = \frac{\phi(c_1-\mu)}{\phi(c_2(c_1)-\mu)}$ for $c_1, c_2 \in S$.

Now, we need to solve the following:
\begin{equation}
  0 = g_\mu(c_1) = \int_{(c_1,c_2(c_1)) \cap S} x \phi(x-\mu)\, dx -  m(\mu)
\end{equation}


Next, exploiting the identity
\begin{align}
  \delta_\mu(a,b) &\triangleq \int_a^b x\phi(x-\mu)\,dx \\
  &= - \phi(b-\mu) + \phi(a-\mu) +
  \mu\left(\Phi(b-\mu)-\Phi(a-\mu)\right),
\end{align}
we can evaluate $g_\mu(c_1)$ with relative ease in terms of sums of $\delta_\mu(a,b)$.

We can also easily evaluate first derivatives of $g_\mu$
for $c_1, c_2(c_1) \in S$:
\begin{align}
  g_\mu'(c_1) &= c_2(c_1) \phi(c_2(c_1) - \mu)c_2'(c_1) - c_1\phi(c_1-\mu)\\
  &= (c_2(c_1)-c_1) \phi(c_1-\mu) > 0
\end{align}

That is, we are just finding a root of a continuous function which is
monotone-increasing for $c_1\in S$.

The easiest way to get confidence intervals is to use bisection,
noting that
\begin{equation}
  g_\mu(x) > 0 \dimp x > c_1(\mu) \dimp \mu_{\text{hi}} > \mu
\end{equation}




\bibliographystyle{plainnat}
%%\bibliography{../../LaTex/biblio}

\end{document}
